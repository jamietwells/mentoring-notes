\documentclass{article}

\usepackage{listings}
\usepackage{color}
\usepackage[most]{tcolorbox}
\lstloadlanguages{C,C++,csh,Java}

\definecolor{identifier}{rgb}{0, 0.68, 0.68}
\definecolor{string}{rgb}{0.75,0,0.2}
\definecolor{comment}{rgb}{0, 0.718, 0.059}
\definecolor{keyword}{rgb}{0.29, 0.59, 0.82}
\definecolor{background}{rgb}{0.07, 0.09, 0.1}
\definecolor{nonKeywords}{rgb}{0.95,0.95,0.95}
\definecolor{captionBox}{rgb}{0.2,0.2,0.2}

\newtcolorbox{titleBox}[2][]{%
  attach boxed title to top center= {yshift=-8pt},
  colback = black,
  colframe = black,
  fonttitle = \bfseries,
  colbacktitle = white,
  coltitle = black,
  title = #2,#1,
  enhanced,
}

\lstset{
language=csh,
basicstyle=\footnotesize\ttfamily\color{nonKeywords},
numbers=left,
numberstyle=\tiny,
numbersep=5pt,
tabsize=2,
extendedchars=true,
breaklines=true,
frame=tb,
stringstyle=\color{string}\ttfamily,
showspaces=false,
showtabs=false,
xleftmargin=17pt,
framexleftmargin=17pt,
framexrightmargin=5pt,
framexbottommargin=4pt,
framextopmargin=4pt,
commentstyle=\color{comment},
morecomment=[l]{//}, %use comment-line-style!
morecomment=[s]{/*}{*/}, %for multiline comments
showstringspaces=false,
morekeywords={ var, abstract, event, new, struct, as, explicit, null, switch, base, extern, object, this, bool, false, operator, throw, break, finally, out, true, byte, fixed, override, try, case,
float, params, typeof, catch, for, private, uint, char, foreach, protected, ulong, checked, goto, public, unchecked, class, if, readonly, unsafe, const, implicit, ref, ushort, continue, in, return,
using, decimal, int, sbyte, virtual, default, interface, sealed, volatile, delegate, internal, short, void, do, is, sizeof, while, double, lock, stackalloc, else, long, static, enum, namespace, string},
keywordstyle=\color{keyword},
identifierstyle=\color{identifier},
backgroundcolor=\color{background},
}

\usepackage{caption}
\DeclareCaptionFont{white}{\color{white}}
\DeclareCaptionFormat{listing}{\colorbox{captionBox}{\parbox{\textwidth}{\hspace{15pt}#1#2#3}}}
\captionsetup[lstlisting]{format=listing,labelfont=white,textfont=white, singlelinecheck=false, margin=0pt, font={bf,footnotesize}}

\begin{document}

\section{Syntax}

\subsection{Task}
Create a program to find all the factors of any positive integer.

\begin{titleBox}[colback=white]{Example}
\begin{verbatim}
Input: 12
Output: 1, 2, 3, 4, 6, 12

Input: 7
Output: 1, 7

Input: 16
Output: 1, 2, 4, 8, 16
\end{verbatim}
\end{titleBox}

\subsection{Task}
Expand the program to calculate the highest common factor between two numbers

\begin{titleBox}[colback=white]{Example}
\begin{verbatim}
Input: 12, 16
Output: 4

Input: 7, 15
Output: 1
\end{verbatim}
\end{titleBox}

\subsection{Notes}
This session focused on becoming comfortable with syntax. You should be comfortable with the correct syntax for classes, methods, fields, and other members.

\begin{lstlisting}[language={[Sharp]C}, title={Creating a class}, label={Script}]
public class MyClass {
   ...
}
\end{lstlisting}

\begin{lstlisting}[language={[Sharp]C}, title={Creating properties}, label={Script}]
public class MyClass {
   public int MyProperty { get; set; } 
}
\end{lstlisting}

\begin{lstlisting}[language={[Sharp]C}, title={Creating fields}, label={Script}]
public class MyClass {
   private int MyField; 
}
\end{lstlisting}

You should also be comfortable calling methods and refering to variables and properties in the code, as well as declaring variables.

\begin{lstlisting}[language={[Sharp]C}, title={Declaring variables}, label={Script}]
public class MyClass {
  public static int AddNumbers(int first, int second) {
    var result = first + second;
    return result;
  }
}
\end{lstlisting}

We also looked at creating Lists and looping through lists.

\begin{lstlisting}[language={[Sharp]C}, title={Creating a list and adding items to it}, label={Script}]
public class MyClass {
  public static List<int> GetNumbersFrom1To5() {
    // Make list
    var numbers = new List<int>();

    // populate list with values
    numbers.Add(1);
    numbers.Add(2);
    numbers.Add(3);
    numbers.Add(4);
    numbers.Add(5);
    
    // return list to the caller of this method
    return numbers;	
  }
}
\end{lstlisting}

\begin{lstlisting}[language={[Sharp]C}, title={Looping over a list}, label={Script}]
public class MyClass {
  public static void PrintAllInList() {
    var numbers = GetNumbersFrom1To5(); // method from above
    foreach(var number in numbers) {
      Console.WriteLine(number);
    }
  }
}
\end{lstlisting}


\section{Object oriented programming}

\subsection{Task}
Create a program to store detals about school courses, and list them all, when requested.

\begin{titleBox}[colback=white]{Example}
\begin{verbatim}
Input: 1

Output: 
1. List all courses
2. Search for student

All Courses:
    Core:
        Astronomy
        Charms
        Defence Against the Dark Arts
        Flying
        Herbology
        History of Magic
        Potions
        Transfiguration

    Optional:
        Alchemy
        Apparition
        Arithmancy
        Care of Magical Creatures
        Divination
        Study of Ancient Runes

    Extra Curricular:
        Advanced Arithmancy Studies
        Ancient Studies
        Magical Theory
        Orchestra
\end{verbatim}
\end{titleBox}

\end{document}